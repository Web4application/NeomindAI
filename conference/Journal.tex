\documentclass[conference]{IEEEtran}

\title{NeomindAI: An AI-Driven Cyber-Physical System for Intelligent Sensing, Prediction, and Action}

\author{
\IEEEauthorblockN{Seriki Yakub (Kubu Lee)}
\IEEEauthorblockA{
Independent Researcher \\
Email: kubu.lee@neomind.ai
}
}

\begin{document}
\maketitle

\begin{abstract}
NeomindAI is an AI-driven cyber-physical system that integrates Arduino-based sensing and actuation with layered artificial intelligence for real-time perception, predictive modeling, and autonomous execution. The system adopts a modular architecture that decouples hardware interaction from intelligence logic, enabling scalable deployment across edge, local, and cloud environments.
\end{abstract}

\begin{IEEEkeywords}
Cyber-Physical Systems, Edge AI, IoT, Embedded Intelligence, Arduino
\end{IEEEkeywords}

\section{Introduction}
Cyber-physical systems combine computation and physical processes through feedback loops. While embedded platforms enable sensing and actuation, they lack adaptive intelligence. NeomindAI bridges this gap.

\section{Architecture Overview}
The system consists of four layers: Physical, Interface, Intelligence, and Execution.

\section{Physical Layer}
Arduino-compatible microcontrollers interface with sensors and actuators for deterministic I/O.

\section{Intelligence Layer}
This layer performs preprocessing, inference, prediction, and decision-making using rule-based or machine learning models.

\section{Execution and Feedback}
AI decisions are validated and executed on hardware with continuous sensor feedback.

\section{Conclusion}
NeomindAI demonstrates a scalable blueprint for intelligent cyber-physical systems.

\bibliographystyle{IEEEtran}
\begin{thebibliography}{99}
\bibitem{cps} E. A. Lee, ``Cyber Physical Systems,'' IEEE, 2008.
\bibitem{arduino} Arduino Documentation.
\end{thebibliography}

\end{document}
